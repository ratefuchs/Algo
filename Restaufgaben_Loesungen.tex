\documentclass{scrartcl}
\usepackage[T1]{fontenc}
\usepackage[utf8]{inputenc}
\usepackage[ngerman]{babel}
\usepackage{amsmath,amssymb}
\usepackage{graphicx}
\usepackage{enumerate}

\newcommand{\rbt}{Rot-Schwarz-Baum }


\begin{document}
\section*{Übriggebliebene Aufgaben: Lösungen}
\begin{enumerate}[(1)]

\item \begin{verbatim}BINOMIAL(n, k)
//sei b[0..n][0..k] ein neues Feld
for i = 0 to n
    for j = 0 to k
        if (j > i)
            b[i][j] = 0;
        else if (j == 0 or i == j)
            b[i][j] = 1;
        else
            b[i][j] = b[i – 1][k – 1] + b[i – 1][k];
return b[n][k];
\end{verbatim}

\item Gegenbeispiel: Knoten $1,2,3,4$ und folgende Kanten mit Gewichten: $c(\{1,2\})=1$, $c(\{2,3\})=1$, $c(\{3,4\})=1$ und $c({1,4})=2$.
Bilde den k\"urzeste-Wege-Baum von $1$ aus. Benutzte Kanten: $\{1,2\}, \{2,3\}, \{1,4\}$ mit Gewicht 4.
Der minimale Spannbaum besteht aus den drei Kanten mit Gewicht 1 (Gesamtgewicht 3).

\item \begin{enumerate}[(a)]
\item Knoten $1,2,3$ und folgende Kanten mit Gewichten: $c(\{1,2\})=3$, $c(\{2,3\})=-2$, und $c(\{1,3\})=2$. Startknoten sei Knoten $1$.
\item Knoten $1,2,3$ und folgende Kanten mit Gewichten: $c(\{1,2\})=1$, $c(\{2,3\})=-1$, und $c(\{1,3\})=9$. Startknoten sei wieder Knoten $1$.
\item bei ungerichteten Graphen gilt: negative Kante = negativer Kreis (Kante beliebig oft vor und zur\"uck).
\end{enumerate}

\item \begin{enumerate}[(a)]
\item Es ist $a=1$, $b=2$, $f(n)=n$ und $\log_b a=\log_2 1=0$. Daher ist $f(n)=n\in \Omega(n^{0+0,5})$ mit $\varepsilon = 0,5$ und $a\cdot f(\frac{n}{b})=\frac{n}{2}\leq \frac{1}{2}\cdot n = c\cdot f(n)$ mit $c=\frac{1}{2}$. Also ist Fall 3 des Mastertheorems anwendbar, es folgt $T(n)\in\Theta(n)$.
\item Es ist $a=2$, $b=2$, $f(n)=1$ und $\log_b a=\log_2 2=1$. Daher ist $f(n)=1\in O(n^{1-0,5})$ mit $\varepsilon=0,5$. Also ist Fall 1 des Mastertheorems anwendbar, es folgt $T(n)\in\Theta(n)$.
\item Es ist $a=1$, $b=2$, $f(n)=1$ und $\log_b a=\log_2 1=0$. Daher ist $f(n)=1\in \Theta(n^0)=\Theta(1)$. Also ist Fall 2 des Mastertheorems anwendbar, es folgt $T(n)\in\Theta(\log n)$.
\end{enumerate}

\item \begin{enumerate}[(a)]
\item Es ist $a=1$, $b=2$, $f(n)=\log(n)$ und $\log_b a=\log_2 1=0$. Nun ist einerseits $\log(n)\notin O(1)$, also sind die F\"alle 1 und 2 nicht anwendbar. Es gilt zwar $\log(n)\in\omega(1)$, aber es existiert kein $\varepsilon >0$ mit $\log(n)\in\Omega(n^{0+\varepsilon})$, also ist es auch nicht Fall 3.
\item Diese Gleichung hat nicht die Form des Mastertheorems, dort ist $a$ eine Konstante.
\item Es gilt $a=1$, $b=2$, $f(n)=n\cdot (2-\cos(n))$ und $\log_b a=\log_2 1=0$. Da $n\leq n\cdot (2-\cos(n))\leq 3n$, ist $n\cdot (2-\cos(n))\in \Omega(n^{0+0,5})$, was Fall 1 und Fall 2 ausschlie"st, aber die erste Bedingung von Fall 3 erf\"ullt. Die Regularit\"atsbedingung $a\cdot f(\frac{n}{b})\leq c\cdot f(n)$ f\"ur ein $c\in[0,1)$ ist aber verletzt. Dies soll hier nur veranschaulicht werden (und nicht exakt bewiesen). W\"ahle $n=2(2k+1)\pi$ mit $k\in\mathbb{N}$ (allerdings keine ganze Zahl).
$$f(n)=2(2k+1)\pi\cdot(2-\cos(2(2k+1)\pi))=2(2k+1)\pi\cdot(2-1)=2(2k+1)\pi$$
$$f(\frac{n}{2})=(2k+1)\pi\cdot(2-\cos((2k+1)\pi))=(2k+1)\pi\cdot(2+1)=3(2k+1)\pi=\frac{3}{2}f(n)$$
Wenn $n$ nahe genug bei einer nat\"urlichen Zahl liegt, wird die Regularit\"atsbedingung auch durch diese verletzt.
\item Diese Gleichung ist \"uberhaupt nicht l\"osbar, denn $T(n)=T(n)+1\Rightarrow 0=1$: Widerspruch!
\end{enumerate}

\item \begin{enumerate}[(a)]
\item Falsch! W\"ahle $f(n)=n, g(n)=2n, h(n)=j(n)=2^n$.
\item Wahr! Wegen $f\in\Theta(g)$ und $h\in\Theta(j)$ folgt: $\exists a,b\in \mathbb{R}^+, N\in\mathbb{N}$ mit $f(n)\leq a\cdot g(n)$ und $h(n)\leq b\cdot j(n)$ f\"ur $n\geq N$. Nun gilt:
$$\underbrace{f(n)}_{\geq 0}\cdot\underbrace{h(n)}_{\geq 0}\leq \underbrace{a\cdot b}_{:=c}\cdot g(n)\cdot j(n)\qquad (n\geq N)$$
Der Beweis f\"ur die andere Ungleichung l\"auft analog.
\item Falsch! W\"ahle $f(n)=g(n)=h(n)=n, j(n)\equiv 1$.
\item Falsch! W\"ahle $f(n)=2^n, g(n)=4^n, h(n)=\lfloor \log_2 n\rfloor$.\newline
Alternative L\"osung: $h(n)\equiv 1$.
\item Falsch! W\"ahle $f(n)=2n, g(n)=n, h(n)=2^n$.
\item Wahr! Wegen $f\in o(g)$ folgt: $\forall a\in\mathbb{R}^+\exists N\in\mathbb{N}: f(n)<a\cdot g(n)$. W\"ahle $a=1\Rightarrow \exists N\in \mathbb{N}:f(n)<g(n)$. Nun gilt aber, da $h$ monoton wachsend:
$$h(f(n))\leq h(g(n))=1\cdot h(g(n))\qquad (n\geq N)$$
\item Wahr! $\forall n\in \mathbb{N}_0: n^2\geq n$. Da $f(n)\in\mathbb{N}_0$, folgt automatisch $f(n)\leq (f(n))^2=1\cdot (f(n))^2$.
\item Falsch! W\"ahle $f(n)\equiv 1$.
\item Falsch! W\"ahle $f(n)=n$.
\end{enumerate}

\item \begin{verbatim}
bh(x,T)
if x = T.wurzel
    return 0
else
    return (x.farbe == schwarz ? 1 : 0) + bh(x.vater, T)
\end{verbatim}

\item L\"osungsskizze: pr\"ufe zuerst, ob Wurzel schwarz. Gehe nun rekursiv bis zu den Bl\"attern. Auf dem Weg dorthin k\"onnen die Eigenschaften "`Jeder Knoten ist entweder rot oder schwarz"' und "`Ein roter Knoten hat schwarze Kinder"' \"uberpr\"uft werden. In den Bl\"attern angekommen, wird getestet ob diese schwarz sind. F\"ur die Schwarz-H\"ohe wird (1) verwendet.

\item \begin{enumerate}[(a)]
\item Schwarzh\"ohe des Baums ist minimal $\lceil\frac{h}{2}\rceil$ und maximal $h$. Die minimal m\"ogliche H\"ohe von Knoten ist damit $\lceil\frac{h}{2}\rceil$. Die Differenz ist also $h-\lceil\frac{h}{2}\rceil=\lfloor\frac{h}{2}\rfloor$.
\item Die L\"osung ist ein Baum mit genau einem roten Knoten (nicht an der Wurzel).
\end{enumerate}

\item \begin{enumerate}[(a)]
\item Blatt-Knoten unter dem roten Knoten haben die H\"ohe $h$. Der Unterbaum des roten Knotens ist vollst\"andig und hat die H\"ohe $h-d$. Andere Blatt-Knoten haben kleinere H\"ohe $\Rightarrow a(h)=2^{h-d}$.\newline
Auf der H\"ohe $d$ gibt es $2^d$ Knoten, davon $2^d-1$ schwarze. Die entsprechende Blatt-Knoten haben H\"ohe $h$, die Unterb\"aume haben also die H\"ohe $h-d-1 \Rightarrow a(h-1)=(2^d-1)\cdot 2^{h-d-1}$.\newline
F\"ur alle anderen $n$ ist $a(n) =0$.
\item $a(h)=2^{h-d}\stackrel{!}{=}2^h\Rightarrow d=0$. Damit ist der rote Knoten die Wurzel. Diese muss aber schwarz sein.
\end{enumerate}

\item Die Anzahl der Hashtabellenslots muss mindestens proportional zur Anzahl der gespeicherten Elemente sein, d.h. im Durchschnitt darf höchstens 1 Element pro Slot gespeichert sein.

\item \begin{enumerate}[(a)]
\item Amortisierte Analyse ist eine Art, die Laufzeit einer Operation zu bestimmen. Es wird der Durchschnitt der Laufzeit über eine (Worst-Case-)Folge von Operationen gebildet. Dabei wird ausgenutzt, dass ein Worst Case zwar eine längere Laufzeit hat, aber so selten auftreten kann, dass er den Durchschnitt praktisch nicht verändert.
\item Einfügen bzw. Löschen in unbeschränkten Feldern. Worst Case: $\Theta(n)$, Amortisierte Analyse: $\Theta(1)$.
\item Gemeinsamkeit: Durchschnittsbildung der Laufzeit über mehrere Operationen\newline
Unterschiede: Amortisierte Analyse ist trotzdem eine Worst-Case-Betrachtung und kein Avereage Case; amortisierte Analyse betrachtet Folgen von Operationen, Average-Case-Analyse jedoch unabhängige Operationen
\end{enumerate}
\end{enumerate}
\end{document}