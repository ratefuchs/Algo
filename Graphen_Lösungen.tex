\documentclass{scrartcl}
\usepackage[T1]{fontenc}
\usepackage[utf8]{inputenc}
\usepackage[ngerman]{babel}
\usepackage{amsmath,amssymb}
\usepackage{enumerate}

\begin{document}
\section*{Lösungen zum Thema Graphen}
\begin{enumerate}[(1)]

\item Knoten $1,2,3,4$ und folgende Kanten mit Gewichten: $c(\{1,2\})=1$, $c(\{2,3\})=1$, $c(\{3,4\})=1$ und $c({1,4})=2$.
Bilde den k\"urzeste-Wege-Baum von $1$ aus. Benutzte Kanten: $\{1,2\}, \{2,3\}, \{1,4\}$ mit Gewicht 4.
Der minimale Spannbaum besteht aus den drei Kanten mit Gewicht 1 (Gesamtgewicht 3).

\item \begin{enumerate}[(a)]
\item Knoten $1,2,3$ und folgende Kanten mit Gewichten: $c(\{1,2\})=3$, $c(\{2,3\})=-2$, und $c(\{1,3\})=2$. Startknoten sei Knoten $1$.
\item Knoten $1,2,3$ und folgende Kanten mit Gewichten: $c(\{1,2\})=1$, $c(\{2,3\})=-1$, und $c(\{1,3\})=9$. Startknoten sei wieder Knoten $1$.
\item bei ungerichteten Graphen gilt: negative Kante = negativer Kreis (Kante beliebig oft vor und zur\"uck).
\end{enumerate}

\end{enumerate}
\end{document}