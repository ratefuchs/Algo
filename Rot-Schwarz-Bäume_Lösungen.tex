\documentclass{scrartcl}
\usepackage[T1]{fontenc}
\usepackage[utf8]{inputenc}
\usepackage[ngerman]{babel}
\usepackage{amsmath,amssymb}
\usepackage{enumerate}

\newcommand{\rbt}{Rot-Schwarz-Baum }

\begin{document}
\section*{Lösungen zum Thema Rot-Schwarz-Bäume}
\begin{enumerate}[(1)]

\item \begin{verbatim}
bh(x,T)
if x = T.wurzel
    return 0
else
    return (x.farbe == schwarz ? 1 : 0) + bh(x.vater, T)
\end{verbatim}

\item L\"osungsskizze: pr\"ufe zuerst, ob Wurzel schwarz. Gehe nun rekursiv bis zu den Bl\"attern. Auf dem Weg dorthin k\"onnen die Eigenschaften "`Jeder Knoten ist entweder rot oder schwarz"' und "`Ein roter Knoten hat schwarze Kinder"' \"uberpr\"uft werden. In den Bl\"attern angekommen, wird getestet ob diese schwarz sind. F\"ur die Schwarz-H\"ohe wird (1) verwendet.

\item \begin{enumerate}[(a)]
\item Schwarzh\"ohe des Baums ist minimal $\lceil\frac{h}{2}\rceil$ und maximal $h$. Die minimal m\"ogliche H\"ohe von Knoten ist damit $\lceil\frac{h}{2}\rceil$. Die Differenz ist also $h-\lceil\frac{h}{2}\rceil=\lfloor\frac{h}{2}\rfloor$.
\item Die L\"osung ist ein Baum mit genau einem roten Knoten (nicht an der Wurzel).
\end{enumerate}

\item \begin{enumerate}[(a)]
\item Blatt-Knoten unter dem roten Knoten haben die H\"ohe $h$. Der Unterbaum des roten Knotens ist vollst\"andig und hat die H\"ohe $h-d$. Andere Blatt-Knoten haben kleinere H\"ohe $\Rightarrow a(h)=2^{h-d}$.\newline
Auf der H\"ohe $d$ gibt es $2^d$ Knoten, davon $2^d-1$ schwarze. Die entsprechende Blatt-Knoten haben H\"ohe $h$, die Unterb\"aume haben also die H\"ohe $h-d-1 \Rightarrow a(h-1)=(2^d-1)\cdot 2^{h-d-1}$.\newline
F\"ur alle anderen $n$ ist $a(n) =0$.
\item $a(h)=2^{h-d}\stackrel{!}{=}2^h\Rightarrow d=0$. Damit ist der rote Knoten die Wurzel. Diese muss aber schwarz sein.
\end{enumerate}

\end{enumerate}
\end{document}