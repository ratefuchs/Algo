\documentclass{scrartcl}
\usepackage[T1]{fontenc}
\usepackage[utf8]{inputenc}
\usepackage[ngerman]{babel}
\usepackage{amsmath,amssymb}
\usepackage{enumerate}

\newcommand{\rbt}{Rot-Schwarz-Baum }

\begin{document}
\section*{Aufgaben zum Thema Rot-Schwarz-Bäume}
\begin{enumerate}[(1)]

\item Schreibe die Methode $bh(x,T)$, die f\"ur jeden Knoten $x$ seine Schwarzh\"ohe zur\"uckgibt.

\item Schreibe eine Methode $check\_b\_r\_tree(T)$, die einen \rbt als Parameter nimmt und \"uberpr\"uft, ob dieser ein g\"ultiger \rbt ist.

\item \begin{enumerate}[(a)]
\item Ein \rbt habe die H\"ohe $h$. Berechne die maximale Differenz zwischen Entfernungen von der Wurzel zu den NIL-Knoten (mit kurzer Begr\"undung).
\item Konstruiere einen solchen \rbt f\"ur $h=3$.
\end{enumerate}

\item \begin{enumerate}[(a)]
\item Gegeben sei ein \rbt der H\"ohe $h$ mit genau einem roten Knoten. Dieser habe den Abstand $d$ von der Wurzel.
Berechne f\"ur jedes $n\in \mathbb{N}$ die Anzahl $a(n)$ der NIL-Knoten, die den Abstand $n$ zur Wurzel haben.
\item Ist es m\"oglich, dass $a(h)=2^h$? Begr\"unde!
\end{enumerate}

\end{enumerate}
\end{document}