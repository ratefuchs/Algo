\documentclass{scrartcl}
\usepackage[T1]{fontenc}
\usepackage[utf8]{inputenc}
\usepackage[ngerman]{babel}
\usepackage{amsmath,amssymb}
\usepackage{enumerate}

\begin{document}
\section*{Aufgaben zum Thema Laufzeit}
\begin{enumerate}[(1)]

\item Gegeben sei ein Datentyp, dessen Elemente bez\"uglich < bzw. > vergleichbar sind. Im folgenden betrachten wir Datenstrukturen f\"ur Elemente dieses Typs.
\begin{enumerate}[(a)]
\item Warum l\"asst sich in einem Heap das Minimum nicht in $\Theta(1)$ entfernen, so dass die entstehende Struktur immer noch ein Heap ist? \newline
Tipp: Analysieren Sie die Laufzeit des sich ergebenden Heapsorts.
\item Geben Sie eine Datenstruktur an, die die entsprechende Eigenschaft aus (a) hat.
\item Warum ist dies kein Widerspruch zur \"Uberlegung aus (a)?
\end{enumerate}

\item L\"osen Sie folgende Rekurrenzgleichungen:
\begin{enumerate}[(a)]
\item $T(n)=T(\frac{n}{2})+n$
\item $T(n)=2\cdot T(\frac{n}{2})+1$
\item $T(n)=T(\frac{n}{2})+1$
\end{enumerate}

\item Warum l\"asst sich das Mastertheorem nicht auf folgende Gleichungen anwenden?
\begin{enumerate}[(a)]
\item $T(n)=T(\frac{n}{2})+\log(n)$
\item $T(n)=n\cdot T(\frac{n}{2})+1$
\item $T(n)=T(\frac{n}{2})+n\cdot(2-\cos(n))$
\item $T(n)=T(n)+1$
\end{enumerate}

\item Seien $f,g,h,j: \mathbb{N}\to\mathbb{N}_0$ monoton wachsend. Zeigen oder widerlegen Sie:
\begin{enumerate}[(a)]
\item $f\in\Theta(g)$ und $h\in\Theta(j)\Rightarrow f\circ h \in \Theta(g\circ j)$
\item $f\in\Theta(g)$ und $h\in\Theta(j)\Rightarrow f\cdot h \in \Theta(g\cdot j)$
\item $f\in\Theta(g)$ und $h\in\omega(j)\Rightarrow f + h \in \Theta(g + j)$
\item $f\in o(g)\Rightarrow h\circ f \in o(h\circ g)$
\item $f\in O(g)\Rightarrow h\circ f \in O(h\circ g)$
\item $f\in o(g)\Rightarrow h\circ f \in O(h\circ g)$
\item $f\in O(f^2)$
\item $f\in o(f^2)$
\item $f\in \Theta(f^2)$
\end{enumerate}

\end{enumerate}
\end{document}