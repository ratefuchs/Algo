\documentclass{scrartcl}
\usepackage[T1]{fontenc}
\usepackage[utf8]{inputenc}
\usepackage[ngerman]{babel}
\usepackage{amsmath,amssymb}
\usepackage{enumerate}
\usepackage{graphicx}

\begin{document}
\section*{L\"osungen zum Thema Sortieren}
\begin{enumerate}[(1)]

\item
\begin{enumerate}[(a)]
\item
\text{ } \\
\begin{center}
\begin{tabular}{|c|c|c|c|c|c|}
\hline
4 & 2 & 1 & 3 & 7 & 3 \\
\hline
2 & 4 & 1 & 3 & 7 & 3 \\
\hline
2 & 1 & 4 & 3 & 7 & 3 \\
\hline
2 & 1 & 3 & 4 & 7 & 3 \\
\hline
2 & 1 & 3 & 4 & 7 & 3 \\
\hline
2 & 1 & 3 & 4 & 3 & 7 \\
\hline
2 & 1 & 3 & 4 & 3 & 7 \\
\hline
2 & 1 & 3 & 3 & 4 & 7 \\
\hline
2 & 1 & 3 & 3 & 4 & 7 \\
\hline
2 & 1 & 3 & 3 & 4 & 7 \\
\hline
1 & 2 & 3 & 3 & 4 & 7 \\
\hline
\end{tabular}
\end{center}
\text{} \\
\item Tja, hier gibt es keine Musterl\"osung! ($\rightarrow$ Wikipedia)
\item Beispielsweise die Arrays, die bereits sortiert sind bis auf das kleinste Element, welches ganz rechts steht.
\text{ } \\
\begin{center}
\begin{tabular}{|c|c|c|c|}
\hline
2 & \ldots & n & 1 \\
\hline
\end{tabular}
\end{center}
\end{enumerate}
\end{enumerate}

\end{document}
