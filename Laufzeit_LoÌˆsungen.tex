\documentclass{scrartcl}
\usepackage[T1]{fontenc}
\usepackage[utf8]{inputenc}
\usepackage[ngerman]{babel}
\usepackage{amsmath,amssymb}
\usepackage{enumerate}

\begin{document}
\section*{Lösungen zum Thema Laufzeit}
\begin{enumerate}[(1)]

\item \begin{enumerate}[(a)]
\item Der Heap l\"asst sich in $\Theta(n)$ aufbauen. Wenn das Extrahieren des Minimums jedes Mal $\Theta(1)$ braucht, dauern alle Extrahierungsaktionen zusammen $\Theta(n)$, da man diesen Schritt $n$ Mal macht. Es ergibt sich eine Gesamtlaufzeit von $\Theta(n)$, was der unteren Schranke des Sortierproblems widerspricht.
\item Eine aufsteigend sortierte verkettete Liste erf\"ullt genau das gew\"unschte (Extrahieren des Minimums ist dann einfach "`remove head"').
\item Beim Aufbau dieser Datenstruktur muss die Liste sortiert werden. Dies braucht wegen der unteren Schranke mindestens $\Theta(n\log n)$. F\"ur ein Sortierverfahren ist der Aufbau der Datenstruktur aber n\"otig. Also widerspricht dies nicht den \"Uberlegungen aus (a).
\end{enumerate}

\item \begin{enumerate}[(a)]
\item Es ist $a=1$, $b=2$, $f(n)=n$ und $\log_b a=\log_2 1=0$. Daher ist $f(n)=n\in \Omega(n^{0+0,5})$ mit $\varepsilon = 0,5$ und $a\cdot f(\frac{n}{b})=\frac{n}{2}\leq \frac{1}{2}\cdot n = c\cdot f(n)$ mit $c=\frac{1}{2}$. Also ist Fall 3 des Mastertheorems anwendbar, es folgt $T(n)\in\Theta(n)$.
\item Es ist $a=2$, $b=2$, $f(n)=1$ und $\log_b a=\log_2 2=1$. Daher ist $f(n)=1\in O(n^{1-0,5})$ mit $\varepsilon=0,5$. Also ist Fall 1 des Mastertheorems anwendbar, es folgt $T(n)\in\Theta(n)$.
\item Es ist $a=1$, $b=2$, $f(n)=1$ und $\log_b a=\log_2 1=0$. Daher ist $f(n)=1\in \Theta(n^0)=\Theta(1)$. Also ist Fall 2 des Mastertheorems anwendbar, es folgt $T(n)\in\Theta(\log n)$.
\end{enumerate}

\item \begin{enumerate}[(a)]
\item Es ist $a=1$, $b=2$, $f(n)=\log(n)$ und $\log_b a=\log_2 1=0$. Nun ist einerseits $\log(n)\notin O(1)$, also sind die F\"alle 1 und 2 nicht anwendbar. Es gilt zwar $\log(n)\in\omega(1)$, aber es existiert kein $\varepsilon >0$ mit $\log(n)\in\Omega(n^{0+\varepsilon})$, also ist es auch nicht Fall 3.
\item Diese Gleichung hat nicht die Form des Mastertheorems, dort ist $a$ eine Konstante.
\item Es gilt $a=1$, $b=2$, $f(n)=n\cdot (2-\cos(n))$ und $\log_b a=\log_2 1=0$. Da $n\leq n\cdot (2-\cos(n))\leq 3n$, ist $n\cdot (2-\cos(n))\in \Omega(n^{0+0,5})$, was Fall 1 und Fall 2 ausschlie"st, aber die erste Bedingung von Fall 3 erf\"ullt. Die Regularit\"atsbedingung $a\cdot f(\frac{n}{b})\leq c\cdot f(n)$ f\"ur ein $c\in[0,1)$ ist aber verletzt. Dies soll hier nur veranschaulicht werden (und nicht exakt bewiesen). W\"ahle $n=2(2k+1)\pi$ mit $k\in\mathbb{N}$ (allerdings keine ganze Zahl).
$$f(n)=2(2k+1)\pi\cdot(2-\cos(2(2k+1)\pi))=2(2k+1)\pi\cdot(2-1)=2(2k+1)\pi$$
$$f(\frac{n}{2})=(2k+1)\pi\cdot(2-\cos((2k+1)\pi))=(2k+1)\pi\cdot(2+1)=3(2k+1)\pi=\frac{3}{2}f(n)$$
Wenn $n$ nahe genug bei einer nat\"urlichen Zahl liegt, wird die Regularit\"atsbedingung auch durch diese verletzt.
\item Diese Gleichung ist \"uberhaupt nicht l\"osbar, denn $T(n)=T(n)+1\Rightarrow 0=1$: Widerspruch!
\end{enumerate}

\item \begin{enumerate}[(a)]
\item Falsch! W\"ahle $f(n)=n, g(n)=2n, h(n)=j(n)=2^n$.
\item Wahr! Wegen $f\in\Theta(g)$ und $h\in\Theta(j)$ folgt: $\exists a,b\in \mathbb{R}^+, N\in\mathbb{N}$ mit $f(n)\leq a\cdot g(n)$ und $h(n)\leq b\cdot j(n)$ f\"ur $n\geq N$. Nun gilt:
$$\underbrace{f(n)}_{\geq 0}\cdot\underbrace{h(n)}_{\geq 0}\leq \underbrace{a\cdot b}_{:=c}\cdot g(n)\cdot j(n)\qquad (n\geq N)$$
Der Beweis f\"ur die andere Ungleichung l\"auft analog.
\item Falsch! W\"ahle $f(n)=g(n)=h(n)=n, j(n)\equiv 1$.
\item Falsch! W\"ahle $f(n)=2^n, g(n)=4^n, h(n)=\lfloor \log_2 n\rfloor$.\newline
Alternative L\"osung: $h(n)\equiv 1$.
\item Falsch! W\"ahle $f(n)=2n, g(n)=n, h(n)=2^n$.
\item Wahr! Wegen $f\in o(g)$ folgt: $\forall a\in\mathbb{R}^+\exists N\in\mathbb{N}: f(n)<a\cdot g(n)$. W\"ahle $a=1\Rightarrow \exists N\in \mathbb{N}:f(n)<g(n)$. Nun gilt aber, da $h$ monoton wachsend:
$$h(f(n))\leq h(g(n))=1\cdot h(g(n))\qquad (n\geq N)$$
\item Wahr! $\forall n\in \mathbb{N}_0: n^2\geq n$. Da $f(n)\in\mathbb{N}_0$, folgt automatisch $f(n)\leq (f(n))^2=1\cdot (f(n))^2$.
\item Falsch! W\"ahle $f(n)\equiv 1$.
\item Falsch! W\"ahle $f(n)=n$.
\end{enumerate}

\end{enumerate}
\end{document}